\documentclass[aspectratio=43]{beamer}
% Theme works only with a 4:3 aspect ratio
\usetheme{CSCS}

\usepackage{tikz}
\usepackage{pgfplots}
\usepackage{pgfplotstable}
\usetikzlibrary{pgfplots.groupplots,spy,patterns}
\usepackage{listings}
\usepackage{color}
\usepackage{tcolorbox}
\usepackage{anyfontsize}
\usepackage{xspace}
\usepackage{graphicx}

% define footer text
\newcommand{\footlinetext}{Introduction to GPUs in HPC}

% Select the image for the title page
\newcommand{\picturetitle}{cscs_images/image5.pdf}

% fonts for maths
\usefonttheme{professionalfonts}
\usefonttheme{serif}

% source code listing
\newcommand{\axpy}{{\ttfamily axpy}\xspace}

% set indent to a more reasonable level (so that itemize can be used in columns)
\setlength{\leftmargini}{20pt}

\DeclareTextFontCommand{\emph}{\bfseries\color{blue!70!black}}

% Please use the predifined colors:
% cscsred, cscsgrey, cscsgreen, cscsblue, cscsbrown, cscspurple, cscsyellow, cscsblack, cscswhite

\author{Ben Cumming, CSCS}
\title{Introduction to GPUs in HPC}
\subtitle{}
\date{\today}

\begin{document}

% TITLE SLIDE
\cscstitle

%++++++++++++++++++++++++++++++++
\cscschapter{2D and 3D Launch Configurations}
%++++++++++++++++++++++++++++++++

%%%%%%%%%%%%%%%%%%%%%%%%%%%%%%%%%%%%%%%%%%%%
\begin{frame}[fragile]{Launch Configuration}
%%%%%%%%%%%%%%%%%%%%%%%%%%%%%%%%%%%%%%%%%%%%
       \begin{itemize}
           \item So far we have used one-dimensional launch configurations:
           \begin{itemize}
               \item Threads in blocks indexed using \lst{threadIdx.x}.
               \item Blocks in a grid indexed using \lst{blockIdx.x}.
           \end{itemize}
           \item Many kernels map naturally onto 2D and 3D indexing:
           \begin{itemize}
               \item e.g. Matrix-matrix operations;
               \item e.g. Stencils.
           \end{itemize}
       \end{itemize}

\end{frame}

%%%%%%%%%%%%%%%%%%%%%%%%%%%%%%%%%%%%%%%%%%%%
\begin{frame}[fragile]{Full Launch Configuration}
%%%%%%%%%%%%%%%%%%%%%%%%%%%%%%%%%%%%%%%%%%%%
        Kernel launch dimensions can be specified with \lst{dim3} structs
        \begin{center}
        \lst{kernel``<<<``dim3 grid_dim, dim3 block_dim``>>>``(...);}
        \end{center}
       \begin{itemize}
           \item \lst{dim3.x}, \lst{dim3.y} and \lst{dim3.z} specify the launch dimensions;
           \item Can be constructed with 1, 2 or 3 dimensions;
           \item Unspecified \lst{dim3} dimensions are set to 1.
       \end{itemize}

   \begin{code}{launch configuration examples}
        \begin{lstlisting}[style=boxcudatiny]
// 1D: 128x1x1 for 128 threads
dim3 a(128);
// 2D: 16x8x1  for 128 threads
dim3 b(16, 8);
// 3D: 16x8x4  for 512 threads
dim3 c(16, 8, 4);
        \end{lstlisting}
   \end{code}

\end{frame}

%%%%%%%%%%%%%%%%%%%%%%%%%%%%%%%%%%%%%%%%%%%%
\begin{frame}[fragile]{}
%%%%%%%%%%%%%%%%%%%%%%%%%%%%%%%%%%%%%%%%%%%%

    The \lst{threadIdx}, \lst{blockDim}, \lst{blockIdx} and \lst{gridDim} can be treated like 3D points via the \lst{.x}, \lst{.y} and \lst{.z} members.
    \begin{code}{matrix addition example}
        \begin{lstlisting}[style=boxcudatiny]
__global__
void MatAdd(float *A, float *B, float *C, int n) {
    int i = blockIdx.x * blockDim.x + threadIdx.x;
    int j = blockIdx.y * blockDim.y + threadIdx.y;
    if(i<n && j<n) {
        auto pos = i + j*n;
        C[pos] = A[pos] + B[pos];
    }
}
int main() {
    // ...
    dim3 threadsPerBlock(16, 16);
    dim3 numBlocks(n / threadsPerBlock.x, n / threadsPerBlock.y);
    MatAdd@<<<@numBlocks, threadsPerBlock@>>>@(A, B, C);
    // ...
}
        \end{lstlisting}
   \end{code}

\end{frame}

%%%%%%%%%%%%%%%%%%%%%%%%%%%%%%%%%%%%%%%%%%%%
\begin{frame}[fragile]{Exercise: Launch Configuration}
%%%%%%%%%%%%%%%%%%%%%%%%%%%%%%%%%%%%%%%%%%%%
    \begin{itemize}
        \item Write the 2D diffusion stencil in \lstterm{diffusion/diffusion2d.cu}
        \item Set up 2D launch configuration in the main loop
        \item Draw a picture of the solution to validate it
        \begin{itemize}
            \item a plotting script is provided for visualizing the results
            \item use a small domain for visualization
        \end{itemize}
    \end{itemize}

    \begin{terminal}{}
        \begin{lstlisting}[style=terminal]
# Build and run after writing code
srun diffusion2d 8 1000000

# Do the plotting
module load daint-gpu
module load PyExtensions/3.6.5.7-CrayGNU-19.10
python plotting.py -s
        \end{lstlisting}
    \end{terminal}
\end{frame}

\end{document}
