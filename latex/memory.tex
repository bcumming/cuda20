\documentclass[aspectratio=43]{beamer}
% Theme works only with a 4:3 aspect ratio
\usetheme{CSCS}

\usepackage{tikz}
\usepackage{pgfplots}
\usepackage{pgfplotstable}
\usetikzlibrary{pgfplots.groupplots,spy,patterns}
\usepackage{listings}
\usepackage{color}
\usepackage{tcolorbox}
\usepackage{anyfontsize}
\usepackage{xspace}
\usepackage{graphicx}

% define footer text
\newcommand{\footlinetext}{Introduction to GPUs in HPC}

% Select the image for the title page
\newcommand{\picturetitle}{cscs_images/image5.pdf}

% fonts for maths
\usefonttheme{professionalfonts}
\usefonttheme{serif}

% source code listing
\newcommand{\axpy}{{\ttfamily axpy}\xspace}

% set indent to a more reasonable level (so that itemize can be used in columns)
\setlength{\leftmargini}{20pt}

\DeclareTextFontCommand{\emph}{\bfseries\color{blue!70!black}}

% Please use the predifined colors:
% cscsred, cscsgrey, cscsgreen, cscsblue, cscsbrown, cscspurple, cscsyellow, cscsblack, cscswhite

\author{Ben Cumming, CSCS}
\title{Working with GPU memory}
\subtitle{}
\date{\today}

\begin{document}

% TITLE SLIDE
\cscstitle

%%%%%%%%%%%%%%%%%%%%%%%%%%%%%%%%%%%%
\begin{frame}[fragile]{Memory on a Piz Daint Node}
%%%%%%%%%%%%%%%%%%%%%%%%%%%%%%%%%%%%
    \begin{center}
        \includegraphics[width=0.9\textwidth]{./images/node.pdf}
    \end{center}
\end{frame}

%%%%%%%%%%%%%%%%%%%%%%%%%%%%%%%%%%%%
\begin{frame}[fragile]{Host and Device Memory Spaces}
%%%%%%%%%%%%%%%%%%%%%%%%%%%%%%%%%%%%
    \begin{itemize}
        \item The GPU has separate memory to the host CPU
            \begin{itemize}
                \item The host CPU has 64 GB of DDR4 \emph{host memory}
                \item The P100 GPU has 16 GB of HBM2 \emph{device memory}
            \end{itemize}
        \item Kernels executing on the GPU only have access to device memory
            \begin{itemize}
                \item Kernel accesses to host memory are copied to GPU memory first over the (slow) PCIe connection.
            \end{itemize}

            \begin{center}
                \begin{tabular}{lcr}
                    \textbf{host $\leftrightarrow$ device} & 11$\times$2 GB/s & PCIe gen3 \\
                    \textbf{host memory}                   & 45 GB/s  & DDR4      \\
                    \textbf{device memory}                 & 558 GB/s & HBM2
                \end{tabular}
            \end{center}

        \item \emph{Optimization tip}: The massive bandwidth of HBM2 on P100 GPUs can only help if data is in the right memory space \emph{before} computation starts.
    \end{itemize}

\end{frame}

%%%%%%%%%%%%%%%%%%%%%%%%%%%%%%%%%%%%
\begin{frame}[fragile]{Unified Memory}
%%%%%%%%%%%%%%%%%%%%%%%%%%%%%%%%%%%%
    \begin{center}
        \includegraphics[width=0.9\textwidth]{./images/node_managed.pdf}
        \\
        CUDA unified memory presents a single memory space that can be accessed by both host and GPU code
    \end{center}
\end{frame}

%%%%%%%%%%%%%%%%%%%%%%%%%%%%%%%%%%%%
\begin{frame}[fragile]{Unified Memory}
%%%%%%%%%%%%%%%%%%%%%%%%%%%%%%%%%%%%
        Unified memory presents a single memory space.
        \begin{itemize}
            \item Both CPU and GPU can access the same memory.
            \item First introduced with CUDA 6 and Keplar.
            \item Improved with CUDA 8 and Pascal:
            \begin{itemize}
                \item All host and device memory can be addressed
                \item The \emph{page migration} engine transfers data between GPU and CPU memory as needed
                \item API provides fine-grained control of page migration
            \end{itemize}
            \item Simplifies memory management for GPU programming
        \end{itemize}
        \begin{info}{Managed memory is useful for porting to the GPU.}
            \begin{itemize}
                \item Not suitable as the default choice of memory management.
                \item Can lead to negative performance and subtle bugs.
            \end{itemize}
        \end{info}
\end{frame}

%%%%%%%%%%%%%%%%%%%%%%%%%%%%%%%%%%%%
\begin{frame}[fragile]{Accessing Memory}
%%%%%%%%%%%%%%%%%%%%%%%%%%%%%%%%%%%%
    CUDA uses C pointers to address memory:
    \begin{center}
        \centering \lst{double* data = // address to either host, device or managed memory}
    \end{center}
        \begin{itemize}
            \item A pointer can hold an address in
            \begin{itemize}
                \item \emph{either} device or host memory
                \item managed memory that can \emph{migrate} between host and device
            \end{itemize}
            \item The \emph{CUDA runtime library} provides functions that can be used to allocate, free and copy managed and device memory.
        \end{itemize}
\end{frame}

%%%%%%%%%%%%%%%%%%%%%%%%%%%%%%%%%%%%
\begin{frame}[fragile]{Managing Managed Memory}
%%%%%%%%%%%%%%%%%%%%%%%%%%%%%%%%%%%%
    \begin{info}{Allocating managed memory}
        \centering \lst{cudaMallocManaged(void** ptr, size_t size, unsigned flags)}
    \begin{itemize}
        \item \lst{size} number of \emph{bytes} to allocate.
        \item \lst{ptr} points to allocated memory on return.
        \item \lst{flags} by default is set to \lst{cudaMemAttachGlobal}.
    \end{itemize}
    \end{info}

    \begin{info}{Freeing managed memory}
        \centering \lst{cudaFree(void* ptr)}
    \end{info}

    \begin{code}{Allocate memory for 100 doubles in managed memory}
%..................................
\begin{lstlisting}[style=boxcuda]
double* v;
auto bytes = 100*sizeof(double);
cudaMallocManaged(&v, bytes); // allocate memory
cudaFree(v);                  // free memory
\end{lstlisting}
%..................................
    \end{code}
\end{frame}

%%%%%%%%%%%%%%%%%%%%%%%%%%%%%%%%%%%%%%%%%%%%
\begin{frame}[fragile]{Exercise: Getting started}
%%%%%%%%%%%%%%%%%%%%%%%%%%%%%%%%%%%%%%%%%%%%
    We have to set up the environment before compiling.
    \begin{lstlisting}[style=terminal]
> module load daint-gpu
> module swap PrgEnv-cray PrgEnv-gnu
> module load cudatoolkit
> gcc --version
gcc (GCC) 6.2.0 20160822 (Cray Inc.)
...

> nvcc --version
nvcc: NVIDIA (R) Cuda compiler driver
...
Cuda compilation tools, release 9.1, V9.1.85
    \end{lstlisting}
\end{frame}

%%%%%%%%%%%%%%%%%%%%%%%%%%%%%%%%%%%%%%%%%%%%
\begin{frame}[fragile]{Exercise: Managed Memory Example}
%%%%%%%%%%%%%%%%%%%%%%%%%%%%%%%%%%%%%%%%%%%%
    \begin{enumerate}
        \item open the files \lst{api/managed.cu} and \lst{api/util.hpp}.
        \item what does \lst{managed.cu} do?
        \begin{itemize}
            \item you can use Google!
        \end{itemize}
        \item run it with 20 and 22
    \begin{terminal}{}
    \begin{lstlisting}[style=terminal]
> cd topics/cuda/practicals/api
> make
> srun ./managed 20
    \end{lstlisting}
    \end{terminal}
        \item does it work?
        \item run the cuda profiler
        \begin{terminal}{}
            \begin{lstlisting}[style=terminal]
> srun nvprof -o managed.nvvp --profile-from-start off -f
    ./managed 25
> nvvp managed.nvvp &
            \end{lstlisting}
        \end{terminal}
    \end{enumerate}
\end{frame}

%%%%%%%%%%%%%%%%%%%%%%%%%%%%%%%%%%%%%%%%%%%%
\begin{frame}[fragile]{Concurrent Host-Device Memory Access}
%%%%%%%%%%%%%%%%%%%%%%%%%%%%%%%%%%%%%%%%%%%%
    \begin{columns}[T]
        \begin{column}{0.6\textwidth}
            %\begin{info}{CPU and GPU are asynchronous}
                The CPU:
                \begin{itemize}
                    \item launches the GPU code \lst{gpu_call}
                    \item executes CPU function \lst{cpu_call}
                \end{itemize}
                The GPU:
                \begin{itemize}
                    \item executes gpu code asynchronously
                \end{itemize}
                The problem:
                \begin{itemize}
                    \item both \lst{cpu_call} and \lst{gpu_call} may try to access the same memory
                \end{itemize}
                The solution:
                \begin{itemize}
                    \item synchronize calls between host and device
                \end{itemize}
            %\end{info}
        \end{column}
        \begin{column}{0.4\textwidth}
            \begin{codecolumn}{application code}
                \begin{lstlisting}[style=boxcudatiny]
gpu_call@<<<@...@>>>@();
cpu_call();
                \end{lstlisting}
            \end{codecolumn}
            \includegraphics[width=\textwidth]{./images/async_simple.pdf}
        \end{column}
    \end{columns}
\end{frame}

%%%%%%%%%%%%%%%%%%%%%%%%%%%%%%%%%%%%
\begin{frame}[fragile]{Concurrent Host-Device Memory Access}
%%%%%%%%%%%%%%%%%%%%%%%%%%%%%%%%%%%%
    The CPU can't access managed memory at the same time a GPU kernel is accessing it:
    \begin{itemize}
        \item Doing so causes a segmentation faults or undefined behavior.
        \item To test for synchronization issues run with an environment variable
        \begin{terminal}{}
            \begin{lstlisting}[style=terminal]
    > export CUDA_LAUNCH_BLOCKING=1
            \end{lstlisting}
        \end{terminal}
    \item The CUDA API function \lst{cudaDeviceSynchronize} can be used to force synchronization.
    \end{itemize}
    \begin{code}{}
        \begin{lstlisting}[style=boxcudatiny]
gpu_call@<<<@...@>>>@();
cudaDeviceSynchronize();
cpu_call();
        \end{lstlisting}
    \end{code}{}
\end{frame}

%%%%%%%%%%%%%%%%%%%%%%%%%%%%%%%%%%%%%%%%%%%%
\begin{frame}[fragile]{Exercise: Managed Memory Debugging}
%%%%%%%%%%%%%%%%%%%%%%%%%%%%%%%%%%%%%%%%%%%%
    \begin{enumerate}
        \item Test if concurrent host-device memory access caused the incorrect results in the \lst{api/managed.cu} example.
        \item Can you fix the issue by adding one \lst{cudaDeviceSynchronize()} call?
        \item does the profile from nvprof look different?
    \end{enumerate}
\end{frame}


%%%%%%%%%%%%%%%%%%%%%%%%%%%%%%%%%%%%
\begin{frame}[fragile]{Allocating Device Memory}
%%%%%%%%%%%%%%%%%%%%%%%%%%%%%%%%%%%%

    It is possible to allocate host and device memory directly
    \begin{itemize}
        \item Explicitly allocate memory on device.
        \begin{itemize}
            \item can't be read from host.
        \end{itemize}
        \item Manually copy data to and from host.
        \item For memory that should always reside on device.
        \item The programmer can optimize memory transfers by hand.
        \begin{itemize}
            \item with effort, you can get the best performance this way.
        \end{itemize}
    \end{itemize}
\end{frame}

%%%%%%%%%%%%%%%%%%%%%%%%%%%%%%%%%%%%
\begin{frame}[fragile]{}
%%%%%%%%%%%%%%%%%%%%%%%%%%%%%%%%%%%%
    \begin{info}{Allocating device memory}
        \centering \lst{cudaMalloc(void** ptr, size_t size)}
    \begin{itemize}
        \item \lst{size} number of \emph{bytes} to allocate
        \item \lst{ptr} points to allocated memory on return
    \end{itemize}
    \end{info}

    \begin{info}{Freeing device memory}
        \centering \lst{cudaFree(void* ptr)}
    \end{info}

    \begin{code}{Allocate memory for 100 doubles on device}
%..................................
        \begin{lstlisting}[style=boxcuda]
double* v; // C pointer that will point to device memory
auto bytes = 100*sizeof(double); // size in bytes!
cudaMalloc(&v, bytes); // allocate memory
cudaFree(v);           // free memory
\end{lstlisting}
%..................................
    \end{code}
\end{frame}

%%%%%%%%%%%%%%%%%%%%%%%%%%%%%%%%%%%%
\begin{frame}[fragile]{}
%%%%%%%%%%%%%%%%%%%%%%%%%%%%%%%%%%%%
    \begin{info}{Perform blocking copy (host waits for copy to finish)}
        \centering \lst{cudaMemcpy(void *dst, void *src, size_t size, cudaMemcpyKind kind)}
    \begin{itemize}
        \item \lst{dst} destination pointer
        \item \lst{src} source pointer
        \item \lst{size} number of \emph{bytes} to copy to \lst{dst}
        \item \lst{kind} enumerated type specifying \emph{direction} of copy:
            \\ one of
            \lst{cudaMemcpyHostToDevice}, \lst{cudaMemcpyDeviceToHost}, \lst{cudaMemcpyDeviceToDevice}, \lst{cudaMemcpyHostToHost}
    \end{itemize}
    \end{info}

    \begin{code}{Copy 100 doubles to device, then back to host}
%..................................
        \begin{lstlisting}[style=boxcuda]
auto size = 100*sizeof(double); // size in bytes
double *v_d;
cudaMalloc(&v_d, size);              // allocate on device
double *v_h = (double*)malloc(size); // allocate on host
cudaMemcpy(v_d, v_h, size, cudaMemcpyHostToDevice);
cudaMemcpy(v_h, v_d, size, cudaMemcpyDeviceToHost);
\end{lstlisting}
%..................................
    \end{code}
\end{frame}

%%%%
\begin{frame}[fragile]{}

    \begin{info}{Errors happen\ldots}
        All API functions return error codes that indicate either:
        \begin{itemize}
            \item success;
            \item an error in the API call;
            \item an error in an earlier asynchronous call.
        \end{itemize}
        The return value is the enum type \lst{cudaError_t}
        \begin{itemize}
            \item e.g. \lst{cudaError_t status = cudaMalloc(&v, 100);}
            \begin{itemize}
                \item status is \{\lst{cudaSuccess}, \lst{cudaErrorMemoryAllocation}\}
            \end{itemize}
        \end{itemize}
    \end{info}

    \begin{info}{Handling errors}
        \centering \lst{const char* cudaGetErrorString(status)}
        \begin{itemize}
            \item returns a string describing status
        \end{itemize}
        \centering \lst{cudaError_t cudaGetLastError()}
        \begin{itemize}
            \item returns the last error
            \item resets status to \lst{cudaSuccess}
        \end{itemize}
    \end{info}

\end{frame}

%%%%
\begin{frame}[fragile]{}

    \begin{code}{Copy 100 doubles to device \emph{with error checking}}
%..................................
        \begin{lstlisting}[style=boxcudatiny]
double *v_d;
auto size = sizeof(double)*100;
double *v_host = (double*)malloc(size);
cudaError_t status;

status = cudaMalloc(&v_d, size);
if(status != cudaSuccess) {
  printf("cuda error : %s\n", cudaGetErrorString(status));
  exit(1);
}

status = cudaMemcpy(v_d, v_h, size, cudaMemcpyHostToDevice);
if(status != cudaSuccess) {
  printf("cuda error : %s\n", cudaGetErrorString(status));
  exit(1);
}
        \end{lstlisting}
%..................................
    \end{code}

    \begin{info}{It is essential to test for errors}
        But it is tedious and obfuscates our source code if it is done in line for every API and kernel call\ldots
    \end{info}
\end{frame}

%%%%%%%%%%%%%%%%%%%%%%%%%%%%%%%%%%%%%%%%%%%%
\begin{frame}[fragile]{Exercise: Device Memory API}
%%%%%%%%%%%%%%%%%%%%%%%%%%%%%%%%%%%%%%%%%%%%
    Open \lst{topics/cuda/practicals/api/util.hpp}
    \begin{enumerate}
        \item what does \lst{cuda_check_status()} do?
        \item look at the template wrappers \lst{malloc_host} \& \lst{malloc_device}
        \begin{itemize}
            \item what do they do?
            \item what are the benefits over using \lst{cudaMalloc} and \lst{free} directly?
            \item do we need corresponding functions for \lst{cudaFree} and \lst{free}?
        \end{itemize}

        \item write a wrapper around \lst{cudaMemcpy} for copying data \texttt{host$\rightarrow$device} \& \texttt{device$\rightarrow$host}
        \begin{itemize}
            \item remember to check for errors!
        \end{itemize}

        \item compile the test and run
        \begin{itemize}
            \item it will pass with no errors on success
        \end{itemize}

    \vspace{-5pt}
\begin{terminal}{}
\begin{lstlisting}[style=terminal]
> make explicit
> srun ./explicit 8
\end{lstlisting}
\end{terminal}
    \end{enumerate}

\end{frame}

%%%%%%%%%%%%%%%%%%%%%%%%%%%%%%%%%%%%%%%%%%%%
\begin{frame}[fragile]{Exercise: Device Memory API}
%%%%%%%%%%%%%%%%%%%%%%%%%%%%%%%%%%%%%%%%%%%%
    \begin{enumerate}
        \item How does performance compare with the managed memory version?
        \item What does the nvprof profile look like?
        \begin{itemize}
            \item contrast with managed memory profile.
        \end{itemize}
    \end{enumerate}

\begin{terminal}{}
\begin{lstlisting}[style=terminal]
> srun nvprof -o explicit.nvvp --profile-from-start off -f
    ./explicit 25
> nvvp explicit.nvvp &
\end{lstlisting}
\end{terminal}

\end{frame}

\end{document}
