\documentclass[aspectratio=43]{beamer}
% Theme works only with a 4:3 aspect ratio
\usetheme{CSCS}

\usepackage{tikz}
\usepackage{pgfplots}
\usepackage{pgfplotstable}
\usetikzlibrary{pgfplots.groupplots,spy,patterns}
\usepackage{listings}
\usepackage{color}
\usepackage{tcolorbox}
\usepackage{anyfontsize}
\usepackage{xspace}
\usepackage{graphicx}

% define footer text
\newcommand{\footlinetext}{Introduction to GPUs in HPC}

% Select the image for the title page
\newcommand{\picturetitle}{cscs_images/image5.pdf}

% fonts for maths
\usefonttheme{professionalfonts}
\usefonttheme{serif}

% source code listing
\newcommand{\axpy}{{\ttfamily axpy}\xspace}

% set indent to a more reasonable level (so that itemize can be used in columns)
\setlength{\leftmargini}{20pt}

\DeclareTextFontCommand{\emph}{\bfseries\color{blue!70!black}}

% Please use the predifined colors:
% cscsred, cscsgrey, cscsgreen, cscsblue, cscsbrown, cscspurple, cscsyellow, cscsblack, cscswhite

\author{Ben Cumming, CSCS}
\title{Introduction to GPUs in HPC}
\subtitle{}
\date{\today}

\begin{document}

% TITLE SLIDE
\cscstitle

%++++++++++++++++++++++++++++++++
\cscschapter{Using GPUs in Your Application}
%++++++++++++++++++++++++++++++++

%%%%%%%%%%%%%%%%%%%%%%%%%%%%%%%%%%%%
\begin{frame}[fragile]{}
%%%%%%%%%%%%%%%%%%%%%%%%%%%%%%%%%%%%
    \centering
    Rule \#1: \emph{don't} develop your own GPU code!
\end{frame}

%%%%%%%%%%%%%%%%%%%%%%%%%%%%%%%%%%%%
\begin{frame}[fragile]{Libraries}
%%%%%%%%%%%%%%%%%%%%%%%%%%%%%%%%%%%%
    There are many open libraries for GPUs.
    \begin{itemize}
        \item \href{https://developer.nvidia.com/cublas}{\textcolor{blue}{cuBLAS}}: Dense linear algebra primitives.
        \item \href{https://developer.nvidia.com/thrust}{\textcolor{blue}{Thrust}}: C++ STL-like algorithms and containers.
        \item \href{https://developer.nvidia.com/thrust}{\textcolor{blue}{cuRAND}} and \href{http://www.thesalmons.org/john/random123/releases/latest/docs/}{\textcolor{blue}{Random123}}: Random numbers.
        \item \href{https://developer.nvidia.com/cufft}{\textcolor{blue}{cuFFT}}: FFT
        \item \href{https://github.com/kokkos/kokkos}{\textcolor{blue}{Kokkos}}: Generic performance portable parallel motifs.
    \end{itemize}
    \dots And many more!

    \vspace{1cm}

    Take some time to investigate what is available before starting.
\end{frame}

%%%%%%%%%%%%%%%%%%%%%%%%%%%%%%%%%%%%
\begin{frame}[fragile]{You are going to write your own code?}
%%%%%%%%%%%%%%%%%%%%%%%%%%%%%%%%%%%%

    \begin{info}{Directives}
        \begin{itemize}
            \item OpenACC and OpenMP define \emph{directives} that can be used to instruct the compiler how to generate GPU code.
            \item In theory the easiest path for porting.
        \end{itemize}
    \end{info}

    \begin{info}{GPU-specific Languages}
        \begin{itemize}
            \item Languages designed for GPU programming.
            \item Maximum flexibility and performance.
            \item For example: CUDA, OpenCL and SYCL.
        \end{itemize}
    \end{info}
\end{frame}

%%%%%%%%%%%%%%%%%%%%%%%%%%%%%%%%%%%%
\begin{frame}[fragile]{Things to consider}
%%%%%%%%%%%%%%%%%%%%%%%%%%%%%%%%%%%%
    Before starting on a GPU implementation, it pays to ask some questions and do some preliminary exploration:
    \begin{enumerate}
        \item Is my program computationally or bandwidth intensive?
        \item Does it have enough parallel work to utilize the GPU?
        \item Must I change algorithms to expose enough parallelelism?
        \item Are there serial bottlenecks that will limit scaling?
        \item Is the pain worth the gain?
    \end{enumerate}

    \begin{itemize}
        \item Questions 1, 2 and 3 will be discussed in this course.
        \item Question 4 will be considered briefly here.
        \item Questions 5 requires answers for 1--4.
    \end{itemize}
\end{frame}

%%%%%%%%%%%%%%%%%%%%%%%%%%%%%%%%%%%%
\begin{frame}[fragile]{Limitations to parallel speedup}
%%%%%%%%%%%%%%%%%%%%%%%%%%%%%%%%%%%%
    \begin{itemize}
        \item
            Parallel speedup is limited by\ \emph{the proportion of serial work} in your code.
        \item
            \emph{Amdahl's law} defines the \emph{maximum possible speedup} when only parts of the code can be parallelised
            \begin{equation*}
                t_n = t_1 \left( p+\frac{(1-p)}{n} \right),
            \end{equation*}
            where $t_n$ is time to solution for $n$ threads and $p\in[0,1]$ is the proportion of sequential code.
        \item The limit on time to solution is $\lim_{n\to\infty}=p t_1$
        \begin{itemize}
            \item e.g. 1\% of serial code gives a maximum 100$\times$ speedup.
        \end{itemize}
    \end{itemize}
\end{frame}

%%%%%%%%%%%%%%%%%%%%%%%%%%%%%%%%%%%%
\begin{frame}[fragile]{Amdahl illustrated}
%%%%%%%%%%%%%%%%%%%%%%%%%%%%%%%%%%%%
\begin{tikzpicture}
    \pgfplotsset{footnotesize}
    \begin{axis}[
        height=0.6\textwidth,
        width=\textwidth,
        xmin=1,xmax=128,
        ymin=0,ymax=128,
        xtick={10,20,30,40,50,60,70,80,90,100,110,120,130},
        %ytick={0,1,2,3,4,5,6,7,8},
        xlabel=threads,
        ylabel=speedup,
        legend style = {at={(0,1)}, anchor=north west},
        every axis y label/.style=
            {at={(ticklabel cs:0.5)},rotate=90,anchor=near ticklabel},
        grid=major]
        \addplot[color=blue,thick]  table[x=n,y=effec50]  {./plots/amdahl.tbl};
        \addplot[color=red,thick]  table[x=n,y=effec80]  {./plots/amdahl.tbl};
        \addplot[color=green!50!black,thick]  table[x=n,y=effec90]  {./plots/amdahl.tbl};
        \addplot[color=purple,thick]  table[x=n,y=effec95]  {./plots/amdahl.tbl};
        \addplot[color=black,thick]  table[x=n,y=effec98]  {./plots/amdahl.tbl};
        \addplot[color=cyan,thick]  table[x=n,y=effec99]  {./plots/amdahl.tbl};
        \addplot[color=black,dashed,thick]  table[x=n,y=effec100]  {./plots/amdahl.tbl};
        \legend{50\%, 80\%, 90\%, 95\%, 98\%, 99\%, perfect};
    \end{axis}
\end{tikzpicture}
\end{frame}

%%%%%%%%%%%%%%%%%%%%%%%%%%%%%%%%%%%%
\begin{frame}[fragile]{CUDA}
%%%%%%%%%%%%%%%%%%%%%%%%%%%%%%%%%%%%
    CUDA is a \emph{parallel computing platform and API}
    \begin{itemize}
        \item For CUDA-enabled Nvidia GPUs.
    \end{itemize}
    We use CUDA as short hand for CUDA C/C++ and API
    \begin{itemize}
        \item CUDA C++ is a \emph{superset} of C++
        \item Adds keywords for writing kernels to run on the GPU.
        \item Adds syntax for launching kernels on the GPU.
    \end{itemize}
    The CUDA toolkit is more than a programming language:
    \begin{itemize}
        \item Runtime API for managing GPU resources and execution.
        \item Tools including profilers and debuggers.
    \end{itemize}
\end{frame}

%%%%%%%%%%%%%%%%%%%%%%%%%%%%%%%%%%%%
\begin{frame}[fragile]{Compiling CUDA}
%%%%%%%%%%%%%%%%%%%%%%%%%%%%%%%%%%%%
    CUDA code is compiled with the \emph{nvcc} compiler driver
    \begin{itemize}
        \item source files have \lstterm{.cu} extension
        \item headers have \lstterm{.h}, \lstterm{.hpp}, \lstterm{.hcu} extension.
    \end{itemize}
    CUDA compilation involves multiple splitting, compilation, preprocessing and merging steps
    \begin{itemize}
        \item nvcc hides this complexity from the user.
        \item It closely mimics the interface of the GNU compiler.
        \item Behind the scenes it:
        \begin{itemize}
            \item uses GCC to compile the code that runs on CPU;
            \item and compiles the GPU code separately.
        \end{itemize}
    \end{itemize}
\end{frame}

%%%%%%%%%%%%%%%%%%%%%%%%%%%%%%%%%%%%
\begin{frame}[fragile]{Compiling CUDA with Clang}
%%%%%%%%%%%%%%%%%%%%%%%%%%%%%%%%%%%%
    Clang now supports compilation of CUDA code, targetting NVIDIA GPUs.
    \begin{itemize}
        \item Performance of the generated code is on par with nvcc.
        \item It is a good idea to test your CUDA code with both Clang and nvcc.
        \item The most recent version of the Cray C++ compiler on Daint is Clang based.
    \end{itemize}
\end{frame}

%%%%%%%%%%%%%%%%%%%%%%%%%%%%%%%%%%%%
\begin{frame}[fragile]{Compiling CUDA}
%%%%%%%%%%%%%%%%%%%%%%%%%%%%%%%%%%%%
    Example CUDA compilation
    \begin{terminal}{}
    \begin{lstlisting}[style=terminal]
> nvcc -arch=sm_60 -lineinfo -O2 -std=c++11 -g -o foo foo.cu
    \end{lstlisting}
    \end{terminal}
    Some flags are for \emph{device} code generation:
    \begin{itemize}
        \item \lst{-arch=sm_60} target GPU architecture (Pascal)
        \item \lst{-lineinfo} debug information for device code.
    \end{itemize}
    Some are for \emph{host}:
    \begin{itemize}
        \item \lst{-g} debug information for host code.
    \end{itemize}
    And some are for both \emph{host and device}:
    \begin{itemize}
        \item \lst{-O2} optimization level
        \item \lst{-std=c++11} target language
        \item \lst{-o foo} name of executable.
    \end{itemize}
\end{frame}

%%%%%%%%%%%%%%%%%%%%%%%%%%%%%%%%%%%%
\begin{frame}[fragile]{Compiling CUDA with Clang}
%%%%%%%%%%%%%%%%%%%%%%%%%%%%%%%%%%%%
    Compilation with Clang uses different gpu-specific options:

    \vspace{20pt}
    \begin{terminal}{}
    \begin{lstlisting}[style=terminal]
> CC -xcuda --cuda-gpu-arch=sm_60 --cuda-path=$CUDA_ROOT
          -O2 -std=c++11 -g -o foo foo.cu
    \end{lstlisting}
    \end{terminal}
    \vspace{20pt}
    Where \lst{CC} is the Cray compiler wrapper on Daint.
\end{frame}

%%%%%%%%%%%%%%%%%%%%%%%%%%%%%%%%%%%%
\begin{frame}[fragile]{Exercise: Getting Started on Piz Daint}
%%%%%%%%%%%%%%%%%%%%%%%%%%%%%%%%%%%%
    In this exercise we will get introduced to Daint and make sure that everybody is set up.

    \begin{terminal}{}
        \begin{lstlisting}[style=terminal]
# log on to daint with your course username & password
> ssh -X <your account name>@daint

# get one node on the course reservation for 60 minutes
> salloc -Cgpu --reservation=summer_school -t60

# go to scratch and get the course material
> cd $SCRATCH/SummerSchool2020
> git pull

# compile and test the demo
> cd topics/cuda/practicals/demos
> cat hello.cu
> module load gcc cudatoolkit
> nvcc -arch=sm_60 hello.cu -o hello
> srun ./hello
        \end{lstlisting}
    \end{terminal}
\end{frame}

\end{document}

